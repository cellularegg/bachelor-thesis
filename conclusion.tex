\chapter{Conclusion}
This paper provides an overview of the topic anomaly detection. It provides a description for key features of data quality, and introduces the topic of data cleaning and data cleansing. Furthermore this paper provides general overview of outlier / anomaly detection approaches. Lastly the threshold based outlier detection is further elaborated.
\par
There are countless methods to detect anomalies in data. There is not a go-to approach that suits all needs. It is required to assess different approaches for different applications, in order to get the best result. This paper should provide an overview of approaches to detect outliers / anomalies. It depends on the use case which method to detect outliers has the highest success rate.  
\todo{Change this! Currently copied from the paper.}
% \section{Advantages and Disadvantages of used Outlier Detection Methods}
% \todo{write}
\chapter{Future Work}
Chapter about which additional approaches could be tested, e.g. setting a maximum gradient for both directions (one for rising and falling values) for each measurement station. \todo{Should I also include this or is this not common for a bachelor thesis?}