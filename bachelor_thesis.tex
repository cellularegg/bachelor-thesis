\documentclass[Bachelor, BIF, english, table]{twbook}

\usepackage[T1]{fontenc}
%%\usepackage[ansinew]{inputenc}
\usepackage[utf8]{inputenc}
% Die nachfolgenden 2 Pakete stellen sonst nicht benötigte Features zur Verfügung
%%\usepackage{blindtext,dtklogos}
\usepackage{blindtext, dtk-logos}
\usepackage{cite}

%-------------------------------- for algorithms
%\usepackage{fullpage}
%\usepackage{times}
%\usepackage{fancyhdr,graphicx,amsmath,amssymb}
%\usepackage[ruled,vlined]{algorithm2e}
%\include{pythonlisting}


\usepackage{algorithm}
\usepackage[noend]{algpseudocode}
%--------------------------------

%-------------------------------- for code highlighting
\usepackage{minted}
%\usepackage{xcolor}
\definecolor{LightGray}{gray}{0.9}
\usemintedstyle{borland}
\definecolor{codegreen}{rgb}{0,0.6,0}
\definecolor{codegray}{rgb}{0.5,0.5,0.5}
\definecolor{codepurple}{rgb}{0.58,0,0.82}
\definecolor{backcolour}{rgb}{0.95,0.95,0.92}
%--------------------------------
\usepackage{physics}


%-------------------------------- for commenting
%--- https://tex.stackexchange.com/questions/9796/how-to-add-todo-notes
\usepackage{xargs}                      % Use more than one optional parameter in a new commands
% \usepackage[pdftex,dvipsnames]{xcolor}
\usepackage[colorinlistoftodos,prependcaption,textsize=tiny]{todonotes}
\newcommandx{\unsure}[2][1=]{\todo[linecolor=red,backgroundcolor=red!25,bordercolor=red,#1]{#2}}
\newcommandx{\change}[2][1=]{\todo[linecolor=blue,backgroundcolor=blue!25,bordercolor=blue,#1]{#2}}
% \newcommandx{\info}[2][1=]{\todo[linecolor=OliveGreen,backgroundcolor=OliveGreen!25,bordercolor=OliveGreen,#1]{#2}}
\newcommandx{\improvement}[2][1=]{\todo[linecolor=Plum,backgroundcolor=Plum!25,bordercolor=Plum,#1]{#2}}
%\newcommandx{\thiswillnotshow}[2][1=]{\todo[disable,#1]{#2}}
%-------------------------------- 


%-------------------------------- for UML
\usepackage[]{pgf-umlcd}

%-------------------------------- 
\usepackage{multirow}
\usepackage{booktabs}
\providecommand*{\listingautorefname}{Listing}

%-------------------------------- for highlighting
\usepackage{soul}

\title{Anomaly detection in data for data cleansing}
% \extratitle{Der Untertietel}
\author{David Zelenay}
\studentnumber{000000000000}
\supervisor{Degree First Name Surname}
\place{Vienna}
\kurzfassung{
    % TODO Add Kurzfassung
    Diese Bachelorarbeit befasst sich mit dem Thema Anomalie-Erkennung, mit einem Fokus auf Ausreißererkennung. Der Einstig bildet ein Überblick welche Merkmale Datenqualität ausmachen. Danach werden einige Methoden zur Erkennung von Ausreißern vorgestellt. Diese Methoden werden dann mit Daten aus der echten Welt getestet. Die Daten sind Wasserstandsdaten von Flüssen, welche verwendet werden, um Fluten rechtzeitig zu erkennen und betroffene Personen zu warnen. Bevor die Modelle mit den Daten getestet werden, wurde noch eine explorative Datenanalyse durchgeführt, um sich einen Überblick über die Daten zu verschaffen. Am Schluss werden die Ergebnisse der verschiedenen Ansätze Ausreißer zu erkennen verglichen und ausgewertet. Um die Ergebnisse verlgeichen zu können wurde eine passende Kennzahl definiert, welche auskunft über die Leistung des Modells gibt.
    \change{Change this}
}
\outline{
    % TODO add Abstract
    % \todo{Change this}
    This bachelor thesis analyzes methods for anomaly detection, with a focus on outlier detection for time series data. In the beginning an overview of the characteristics that make up data quality are outlined. Afterwards some methods for outlier detection are presented. These Methods are then tested with real world data. The data used for testing consists of the water levels of different rivers, this data is originally used to warn people about upcoming floods in their area of interest. However before testing the models with the data, an explorative data analysis was conducted to provide an overview of the data. At the end, the results of the different approaches to detect outliers are compared and evaluated. In order to be able to compare the models, a suitable performance metric was defined.
    
    \change{change this}
    % This paper analyzes methods for anomaly detection to cleanse data. An overview of some key features of data quality is provided in the beginning. The difference between data cleaning and data cleansing is elaborated. Additionally a few methods for anomaly detection (mainly outlier detection) are outlined. The hypothesis of this paper is: Which characteristics define data quality, with regard to \ac{IoT} Sensors? Which methods are there to detect and clean or cleanse faulty data?
}
\begin{document}
\maketitle
% \chapter{Planned Structure}
This chapter describes a rough estimate of the planned structure of my bachelor thesis.
\section{Overview}
This section will provide an overview and introduction to the topic anomaly detection and data cleaning/cleansing.
\subsection{Introduction}
Partly already written in the paper. See \autoref{chapter:introduction}.
\subsection{Data cleaning \& data cleansing}
\autoref{chapter:data-cleaning-cleansing-approaches}
(including data cleaning vs data cleansing)
\section{Outlier detection}
\subsection{General Overview}
\autoref{chapter:outlier-detection}

\subsection{Outlier detection methods}
\autoref{section:outlier-detection-approaches}
\subsection{Threshold based outlier detection}
\autoref{section:threshold-based-outlier-detection}
\subsection{Other variants of outlier detection methods}
Provide a deeper insight in other outlier detection approaches. E.g. Clustering / predictive or distance based.
\section{Outlier detection based on "real world data" (pegelalarm.at)}
Create a connection between the theoretical descriptions of outlier detections to a real world use case. Data from \url{https://pegelalarm.at/}
\section{What's the goal?}
Describe the goal to archive: \newline
``
\begin{enumerate}
    \item We are looking for an algorithm that detects outliers using only historical values. This would allow us to assign a probability to the last measured water level of a station, which would indicate how likely it is to be an outlier. We would then not store outliers in our system at all or classify them as outliers from the beginning.
    \item For us also an algorithm would be helpful, which assigns an outlier probability to each arbitrary measured value X of a time series. This algorithm would not only have access to the measured values before it, but also to those after it. This would allow us to detect outliers for all the time series data that we already have in the system and, for example, delete them.

\end{enumerate}
Point 2 is probably easier to implement than point 1, so an algorithm 1 would be more helpful for us. 
Also important would be that the algorithm adjusts its (hyper)parameters accordingly based on the historical data. This means that a level at which there are often strong fluctuations, an outlier must already be quite outlier so that it is considered as an outlier.
''
\subsection{How to retrieve the data (description of the API)}
Short overview on how to use the API to retrieve data? Python project to retrieve data: \url{https://github.com/SOBOS-GmbH/pegelalarm_public_pas_doc}
\subsection{Overview of the data}
Provide an overview oth the data.
\subsection{Explorative data analysis}
Similar to overview of the data
\subsection{Manually detect outliers for a subset of data}
Show cases of outliers in the data and manually classify them. (Also define a way/data structure to classify outliers for time series data)
\subsection{Define outlier detection performance metrics given on a subset of data}
Define a way to compare different outlier detection models / define performance metrics. E.g. number correct outliers, average confidence for the correct outliers, number of missed outliers,.... 
\subsection{Implement different outlier detection approaches}
Develop different outlier detection methods in Python and calculate performance metrics for each
\subsection{Compare different outlier detection approaches}
Compare detection methods from the previous section.
\section{Conclusion}
Summary and conclusion
\subsection{Advantages and disadvantages of used outlier detection methods}

% \showthe\textwidth
% 418.25555pt 
\change{Change title to: ``Anomaly detection for time series data''?}
\chapter{Overview}
\section{Introduction}\label{section:introduction}
With the growing popularity of \ac{IoT} and digitizing business processes, there is a growing amount of data available for analysis.
In order to utilize the data from the \ac{IoT} sensors, it needs to be preprocessed. One step of preprocessing is data cleaning (also referred to as data cleansing).
The main goal of data cleansing is to increase the data quality and furthermore to detect and remove anomalies in the data. The quality requirements for the data can differ depending on the use case. Anomalies in sensor data are data points which do not picture the reality. For example, an anomaly of a temperature sensor would be if the sensor reads 0 $^{\circ}$C and the real temperature is 23 $^{\circ}$C. This paper specifically focuses on the removal of outliers for water level sensors. The data is provided by FloodAlert\cite{strassmayrFloodAlertWaterLevels}, which provides a service to warn people about floods, in their area of interest.
% \todo{Update Introduction}

\section{Research Question}
The research questions are:
\begin{itemize}
    \item What are common methods to detect outliers for time series data?
    \item How can outlier detection methods be compared, with a focus on water level data?
    \item Based on water levels from different rivers, which method is able to classify outliers most reliably?
\end{itemize}
% What are common methods to detect outliers for time series data? \newline
% How can outlier detection methods be compared, with a focus on water level data? \newline
% Based on water levels from different rivers, which method is able to classify outliers most reliably?
% \change{Is this ok?}


\section{Research Method}
This thesis will provide an overview and comparison of different approaches to detect outliers. It will focus on time series data, especially water level measurements of rivers. To introduce the topic a general overview of data quality, data cleansing, data cleaning and outlier types is provided. For the theoretical pars of the chapters, literature research was conducted. After gathering knowledge on different outlier detection approaches they were implemented in Python. To test the performance a suitable performance metric needs to be chosen. To use real-world data to classify outliers, the water levels from different measurement stations were taken. In the end, the performance of the different approaches to detect outliers is compared.
% This paper is a literature research. To get an overview of the topic, papers related to: data cleaning, anomaly detection for IoT data / time series data and outlier detection methods were researched. After some base knowledge was established the major topics of the paper were defined. Subsequently more research was done on the major topics (Features of data quality, data cleaning \& cleansing, outlier detection). To organize the references found while researching Zotero was used, with the Add-on Better BibTeX. \todo{Adapt research method}
 
\chapter{Data Quality}

\section{Features of data quality}\label{section:data-quality-features}
This section will provide a few example key features of data quality. 
\subsubsection{Completeness}
Data completeness describes the wholeness of data. If there are certain aspects of data missing the data is not complete. For example if each datapoint of a sensor includes the date, time and production speed, the data is not complete, if one of those features is missing or not entire, this datapoint is not complete. \cite{caiChallengesDataQuality2015, songIoTDataQuality2020}
\subsubsection{Accuracy}
The accuracy of data describes the exactness. Example for possible data which decrease the accuracy are outliers or time shifts. Usually the accuracy of data is harder to measure than the completeness, consistency, structure or documentation. Due to the heterogeneity of sensor data (regarding numerical values like production speed or temperature, not categorical values like on/off) for each datapoint it is difficult to detect which values are genuine and which are sensor errors and therefore outliers. \cite{caiChallengesDataQuality2015}
\subsubsection{Consistency}
One example for consistency would be, if the data interval is equal. For example there should be a datapoint every ten seconds. As soon as two datapoints are more than ten seconds apart from each other the data is not consistent anymore. \cite{caiChallengesDataQuality2015}
\subsubsection{Structure \& Documentation}
If the structure of the data is not homogeneous, it is very difficult to analyze in an automated way. As a result the data either needs to be structured from the beginning or a process needs to be fabricated to structure the data automatically. Furthermore documentation is required in order to structure and preprocess data. Documentation of data might include data format (\ac{CSV}, parquet \cite{ApacheParquet2021}, \ac{JSON}), date format (e.g. ISO 8601 with UTC offset), valid value spans (e.g. temperature is only valid if it is between 100 and 400 $^{\circ}$C)
\cite{caiChallengesDataQuality2015}

\section{Improving data quality}\label{section:imrpoving-data-quality}
This section will describe methods to improve data quality, based on the features elaborated in \autoref{section:data-quality-features}.
\subsubsection{Completeness}
The most common methods to increase data completeness are statistical and deep learning based approaches. The goal of these methods are to fill in the missing values of a dataset. An example for a statistical method is DynaMMo\cite{liDynaMMoMiningSummarization2009}. For ANNs (artificial neural networks) \ac{LSTM} (Long short-term memory) or \ac{GRU} (Gated recurrent unit) can be used to predict missing data. \cite{songIoTDataQuality2020}
\subsubsection{Accuracy}
One approach to increase the accuracy of data is to define constraints for each value. E.g. When a machine cannot produce more than ten pieces per second, because it is physically not possible, the value could be limited to less or equal than ten. However limiting the values to a specific range might hide the fact that the machine  has an error and is producing faulty products at a rate of 15 pieces per second. This is one of the reasons why more sophisticated outlier detection methods are used. \cite{songIoTDataQuality2020}

\subsubsection{Consistency}
To facilitate consistent data, statistical smoothing or forecasting methods can be used. Examples methods are: \ac{ARIMA} (Autoregressive integrated moving average) or \ac{GP} ( Gaussian Process). ANNs can also be used to unify the time series interval between datapoints. \cite{songIoTDataQuality2020}

\subsubsection{Structure \& Documentation}
The process of structuring heterogeneous and messy data is called data wrangling. In order to unify the structure of the data at least some documentation is required. Therefore the documentation of the data is fundamental in order to analyse or further process it.  
\chapter{Outlier Detection}\label{chapter:outlier-detection}
\section{Outlier Types}\label{section:outlier-detection}
Outliers can be categorized as point outliers or subsequence outliers.
\subsubsection{Point Outliers}
A point outlier is a single datapoint that strongly varies from the usual trend of the datapoints. \cite{blazquez-garciaReviewOutlierAnomaly2020} Examples of three point outliers are shown in \autoref{figure:point-outliers}. \hl{Point outliers are usually easier to detect than subsequence outliers. } \change{Lassen sich diese "outliers" algorithmisch erfassen/detektieren?}
\begin{figure}[h]
  \centering
  \includegraphics{./plots/pdfs/point_outliers.pdf}
  \caption{Examples of three point outliers}
  \label{figure:point-outliers}
\end{figure}

\subsubsection{Subsequence outliers}
Subsequence outliers are multiple consecutive datapoints that strongly vary from the usual trend of the datapoints. \cite{blazquez-garciaReviewOutlierAnomaly2020} In \autoref{figure:subsequence-outliers} examples of subsequence outliers are shown.
\newline\newline
\begin{figure}[h]
  \centering
  \includegraphics{./plots/pdfs/subsequence_outliers.pdf}
  \caption{Examples of three subsequence outliers}
  \label{figure:subsequence-outliers}
\end{figure}
Furthermore outliers can be divided into local and global outliers. 
\subsubsection{Local Outliers}
A local outlier has a greater variance to its direct neighbouring datapoints (previous and next one) \cite{blazquez-garciaReviewOutlierAnomaly2020} In \autoref{figure:local-point-outliers} and \autoref{figure:local-subsequence-outliers} examples of local outliers are shown. The first two outliers in \autoref{figure:local-point-outliers} have a great variance towards it direct neighbours, however not to the values from Time 60 onwards.
\begin{figure}[H]
  \centering
  \includegraphics{./plots/pdfs/local_point_outliers.pdf}
  \caption{Examples of four local point outliers}
  \label{figure:local-point-outliers}
\end{figure}
\begin{figure}[H]
  \centering
  \includegraphics{./plots/pdfs/local_subsequence_outliers.pdf}
  \caption{Examples of six local subsequence outliers}
  \label{figure:local-subsequence-outliers}
\end{figure}
\subsubsection{Global Outliers}
Whereas a global outlier varies more in regard to all datapoints. \autoref{figure:point-outliers}, \ref{figure:subsequence-outliers} and \ref{figure:global-point-outliers} show picture examples of global outliers.
\cite{blazquez-garciaReviewOutlierAnomaly2020}
\begin{figure}[H]
  \centering
  \includegraphics{./plots/pdfs/global_point_outliers.pdf}
  \caption{Examples of four global point outliers}
  \label{figure:global-point-outliers}
\end{figure}

\section{Outlier Detection Approaches}\label{section:outlier-detection-approaches}
Outlier detection methods can be divided into the following groups
\subsubsection{Statistical}
For statistical outlier detection, historical data is taken to develop a model that pictures the expected behavior of the data. An example of a statistical outlier detection is the threshold based method described in \autoref{section:threshold-based-outlier-detection} \cite{cookAnomalyDetectionIoT2020, giannoniAnomalyDetectionModels2018}

\subsubsection{Distance based}
For this approach a distance metric needs to be defined, (e.g. Euclidean distance). Then each datapoint is compared to the data preceding it. The greater the distance between the current and previous datapoints the greater the probability of an anomaly. \cite{cookAnomalyDetectionIoT2020, giannoniAnomalyDetectionModels2018, chandolaAnomalyDetectionSurvey2009}

\subsubsection{Clustering}
Clustering also requires a set of historical data in order to train the clustering model. Usually the data is clustered into two clusters: normal data and anomalous data. Depending on the distance of a new datapoint to the "normal" and the "anomalous" cluster it is classified.
\cite{cookAnomalyDetectionIoT2020, giannoniAnomalyDetectionModels2018, chandolaAnomalyDetectionSurvey2009}
\subsubsection{Predictive}
In this approach a prediction model needs to be developed, based on previous data. The prediction of this model is then compared with the actual datapoint (new data, which was not used in training the model). If the actual datapoint differs too much from the prediction it is labelled as an anomaly. \cite{cookAnomalyDetectionIoT2020, giannoniAnomalyDetectionModels2018}


\subsubsection{Ensemble}
as the word ensemble suggests, this is a collection of outlier detection methods that use a specific vote mechanism to determine whether a datapoint is faulty or normal. For example using the majority vote system and a statistical, distance based and predictive method to detect outliers. If at least two methods flag a datapoint as an outlier the ensemble reports it as an outlier as well. If only one method reports it as an outlier the ensemble does not flag it as an anomaly.
\cite{cookAnomalyDetectionIoT2020}

\section{Threshold based Outlier Detection}\label{section:threshold-based-outlier-detection}
Threshold based detection methods are able to identify outliers based on a given threshold $\tau$. These Methods can be described with the following formula
\begin{equation}
  |x_t - \hat{x}_t | > \tau \text{ \cite{blazquez-garciaReviewOutlierAnomaly2020}}
\end{equation}

Where $x_t$ is the actual value and $\hat{x}_t$ is the expected value and $\tau$ is a given threshold.\\
Methods to calculate $\hat{x}_t$ will be described in the following sections. Furthermore $\hat{x}_t$ can be calculated using the entire data series or with subsets (of equal length) of the entire data series. This means $\hat{x}_t$ can be either calculated for the whole data series or for just a segment.\\
Depending on the sensitivity wanted for outlier detection an appropriate $\tau$ needs to be chosen. The greater $\tau$ is the fewer outliers will be detected. The smaller $\tau$ is the more outliers will be identified.  \cite{blazquez-garciaReviewOutlierAnomaly2020}

\subsubsection{Mean}
\begin{equation}
  \text{mean} = \bar{x} = \frac{1}{n} \sum^n_{t=0}x_t
\end{equation}

Where $n$ is the total number of samples. Using the mean as an expected value is not robust to outliers, because the median is not as robust as the mean in hindsight to outliers. To calculate the mean all datapoints of a series must be summed up and then divided by the number of datapoints.
\subsubsection{Median}
If $n$ is odd:
\begin{equation}
  median(x) = x_{(n+1)/2}
\end{equation}

If $n$ is even:
\begin{equation}
  median(x) = \frac{x_{n/2} + x_{(n+1)/2}}{2}
\end{equation}

Where $x$ is a dataset of $n$ elements ordered from smallest to largest\\
($x_1 \leq x_2 \leq x_3 \leq \ldots \leq x_{n-2} \leq x_{n-1} \leq x_n$)
\cite{blazquez-garciaReviewOutlierAnomaly2020}
To calculate the median all values must be sorted from smallest to largest. If the number of datapoints is odd then the most center datapoint is the Median (e.g. if the series consist of 7 values the third value is the median). If the number of datapoints is even then the median is the mean of the two datapoints in the center.
\subsubsection{\ac{MAD}}
The Median Absolute Deviation 
\begin{equation}
  MAD = median(|x_t - median(x)|)
\end{equation}

$MAD$ is a more robust (regarding outliers) way to calculate the deviation of a dataset. To calculate the $MAD$ firstly the median of the dataset must be calculated. Then the absolute difference between $x_t$ and the median of the dataset is calculated. The Median of all differences results in the $MAD$
\cite{leysDetectingOutliersNot2013, mehrangOutlierDetectionWeight2015}

\section{Outlier Detection using z-score}
\label{section:outlier-detection-z-score}
The z-score, also known as the standard score, is the factor of how many standard deviations a datapoint differs from the mean. Using standard score to detect outliers works best, when the data is distributed normally. Because then it can be assumed, that e.g. the top and bottom 0.5\% are outliers and therefore every value with $|z| > \approx 2.576$ can be classified as an outlier. %2.5758293035489
A visual representation of the z-score in a normal distribution is shown in \autoref{figure:normal-distribution}
\begin{equation}
  z_t = \frac{x_t - \mu}{\sigma}
\end{equation}

Where $\mu$ is the mean of the dataset and $\sigma$ is the standard deviation. When the mean and the standard deviation of the entire dataset is now known the mean and the standard deviation of a known sample can be used.
\cite{DetectionSpatialOutlier, teschlSpezielleStetigeVerteilungen2014, rousseeuwAnomalyDetectionRobust2018}
\begin{figure}[H]
  \centering
  \includegraphics[width=\textwidth]{./pics/The_Normal_Distribution.pdf}
  \caption{Standard Score in a normal distribution\cite{StandardScore2022}}
  \label{figure:normal-distribution}
\end{figure}
\begin{equation}
  \sigma = s = \sqrt{\frac{1}{n-1}\sum^n_{i=1}{(x_i - \bar{x})^2}}
\end{equation}
\cite{teschlSpezielleStetigeVerteilungen2014, rousseeuwAnomalyDetectionRobust2018}
\par
To check whether the value $x_t$ is an outlier the absolute of its z-score is compared against a threshold ($\tau$) and if the absolute value of the z-score exceeds the threshold $x_t$ is classified as an outlier.
\begin{equation}
  |z_t| > \tau
\end{equation}
\section{Outlier Detection using modified z-score}
Because the mean and standard deviation are not robust towards outliers the z-score can be modified to use more robust metrics for the expected value and the variation of the values. To make the outlier detection with the z-score more robust the mean can be replaced with the median and the standard deviation with the \ac{MAD} or the \ac{MADN}. A possible formula for a modified z-score, as described in \cite{baeOutlierDetectionSmoothing2019}, could be:
\begin{equation}
  m_t = \frac{|x_t - median(X)|}{MADN(X)}
\end{equation}
Where the formula for the \ac{MADN} is:
\begin{equation}
  MADN(X) = \frac{MAD(X)}{0.6745}
\end{equation}
The constant value 0.6745 is the 75\textsuperscript{th} percentile of a standard normal distribution, which is equal to the \ac{MAD} of a standard normal distribution with $\sigma = 1$.
\cite{baeOutlierDetectionSmoothing2019}
% http://www.iceaaonline.com/ready/wp-content/uploads/2016/10/RA02-paper-Jones-Outing-Outliers.pdf
\par
The classification of outliers using the modified z-score ($m_t$) is the same for the regular standard score described in \autoref{section:outlier-detection-z-score}:
\begin{equation}
  |m_t| > \tau
\end{equation}


% \section{Other Variants of Outlier Detection Methods}
% Provide a deeper insight in other outlier detection approaches. E.g. Clustering / predictive or distance based.
% \todo{Research other outlier detection methods}
\chapter{Outlier Detection based on Water level Data}
Create a connection between the theoretical descriptions of outlier detections to a real world use case. Data from \url{https://pegelalarm.at/}
\section{What is the Goal?}
Describe the goal to archive: \newline
\begin{itshape}
    ``
    \begin{enumerate}
        \item We are looking for an algorithm that detects outliers using only historical values. This would allow us to assign a probability to the last measured water level of a station, which would indicate how likely it is to be an outlier. We would then not store outliers in our system at all or classify them as outliers from the beginning.
        \item For us also an algorithm would be helpful, which assigns an outlier probability to each arbitrary measured value X of a time series. This algorithm would not only have access to the measured values before it, but also to those after it. This would allow us to detect outliers for all the time series data that we already have in the system and, for example, delete them.

    \end{enumerate}
    Point 2 is probably easier to implement than point 1, so an algorithm 1 would be more helpful for us.
    Also important would be that the algorithm adjusts its (hyper)parameters accordingly based on the historical data. This means that a level at which there are often strong fluctuations, an outlier must already be quite outlier so that it is considered as an outlier.
    ''
\end{itshape}

\section{How to retrieve the Data (Description of the API)}
Short overview on how to use the API to retrieve data? Python project to retrieve data: \url{https://github.com/SOBOS-GmbH/pegelalarm_public_pas_doc}
\section{Overview of the Data}
Provide an overview oth the data.
\section{Explorative Data Analysis}
Similar to overview of the data
\section{Manually detect outliers for a subset of data}
Show cases of outliers in the data and manually classify them. (Also define a way/data structure to classify outliers for time series data)
\newline
\newline
In order to speed up the manual labeling of outliers a program was written. The program is a Plotly \todo{add reference} Dash web application which displays the water level data as a scatter chart. 
By clicking the datapoints in the chart the user is able to toggle the datapoint as an outlier or back to a regular value. 
In Listing \ref{listing:manual-outlier-selection} the source code of the Dash application is shown. 
In \ref{figure:manual-outlier-selection} you can see the Website of the Python app. Below the chart a rangeslider is located, to move the zoomed in view horizontally. 
The refresh button on the left side refreshes the graph, thus updating the color and label of previously selected outliers. Furthermore it saves the data to a parquet file.
The upper and lower limit of the y-axis also gets updated, when pressing the refresh button. 
The limits are automatically set to the lowest and highest regular value. 
This was implemented, because some datasets had outliers with a huge difference towards the regular data. 
Without the automatic scaling of the y-axis, detecting other outliers with a smaller difference was not possible.
% \begin{center}
%     \makebox[\textwidth]{\includegraphics[width=\paperwidth]{...}}
%   \end{center}
\begin{figure}[H]
    \centering
    \includegraphics[width=\textwidth]{./pics/manual-outlier-selection.png}
    \caption{Dash Webapp to classify outliers}
    \label{figure:manual-outlier-selection}
\end{figure}

% https://tex.stackexchange.com/questions/368864/pagebreak-for-minted-in-figure
% \begin{listing}
% \begin{minted}[frame=single]{python}
% print('hw')
% \end{minted}
% \caption{Code example with simple formatting}
% \label{code:manual-outlier-selection}
% \end{listing}

\subsubsection{Manual Outlier Detection Webapp using Dash}
\inputminted[linenos]{python}{./code/manual_outlier_detection.py}
\captionof{listing}{Manual Outlier Detection Webapp using Dash\label{listing:manual-outlier-selection}}
\todo{add comments to code}

\section{Define Outlier Detection performance Metrics given on a Subset of Data}
Define a way to compare different outlier detection models / define performance metrics. E.g. number correct outliers, average confidence for the correct outliers, number of missed outliers,....
\section{Implement different Outlier Detection Approaches}
Develop different outlier detection methods in Python and calculate performance metrics for each
\section{Compare different Outlier Detection Approaches}
Compare detection methods from the previous section.


\chapter{Conclusion}
This bachelor thesis provides an overview of the topic anomaly detection, especially on outlier detection for time series data. It provides a description for key features of data quality, and introduces the topic of data cleaning and data cleansing. Furthermore this paper provides general overview of outlier detection approaches. After a theoretical overview of different outlier detection approaches they are tested on water level data from different rivers.
\par
% There are countless methods to detect anomalies in data. There is not a go-to approach that suits all needs. It is required to assess different approaches for different applications, in order to get the best result. This paper should provide an overview of approaches to detect outliers / anomalies. It depends on the use case which method to detect outliers has the highest success rate.  
The overall bes performance, across different water level measurement stations, was achieved by using the median threshold based outlier detection method with a centered window with the size of three and a threshold of about 6.6. The median threshold based outlier detection also delivered the hightest $F_1-score\;(0.905)$. Using the mean to calculate $\hat{x}_t$ is not recommended since the mean is not robust against outliers. Using the \ac{MAD} with the threshold based outlier detection resulted in the lowest $F_1-score$, with the best score only being about 0.45. The second best result was achieved by using the modified z-score described in \autoref{section:outlier-detection-modified-z-score}. For the stations tested the approach using the median delivered the best performances. However this does not mean, that this will be true for all stations. It has to be assessed for each station individually which model is able to detect best. Furthermore it depends on the use case if higher precision or recall is required. Depending on that, $\beta$ for the $F_{\beta}-score$ needs to be chosen accordingly. For the tests the $F_1-score$ was used since precision and recall are equally important. In addition preprocessing the data by setting upper and lower boundaries and removing datapoints which exceed those limits, did not improve the performance of the models, on the contrary the performance was worse. Because the extreme outliers were mostly detected anyways, thus fewer outliers were detected when setting upper and lower limits, which resulted in a lower performance.

\change{Ich schlage hier vor auch einen Teil der quanitativen Analyse zusammenzufassen. Dabei sollten die Methoden und die Ergebnisse kurz skizziert werden. Vielleicht lassen sich auch Entscheidungen ableiten wann welche Methode besser greift.}
% \todo{Change this! Currently copied from the paper.}
% \section{Advantages and Disadvantages of used Outlier Detection Methods}
% \todo{write}
\chapter{Future Work}
Chapter about which additional approaches could be tested:
\begin{itemize}
    \item setting a maximum gradient for both directions (one for rising and falling values) for each measurement station.
    \item \acp{ANN} with \ac{LSTM} or \ac{GRU} maybe also autoencoder architecture
    \item Prediction vs classification \ac{ANN}
    \item 1.5 times the \ac{IQR}
\end{itemize} 
\todo{Should I also include this or is this not common for a bachelor thesis?}
\change{Die Methode wollte ich auch vorschlagen, aber der Umfang ist so schon gross genug.}
\chapter{Appendix}
Basically include the whole repo: \url{https://github.com/cellularegg/bachelor-thesis-code}
% \section{Code}
\section{Python Code}
\subsection{Manual outlier detection}
{\scriptsize
\inputminted[linenos]{python}{./code/manual_outlier_detection.py}
\captionof{listing}{Manual Outlier Detection Webapp using Dash \cite{zelenayOutlierDetectionWater2022}\label{listing:manual-outlier-selection}}}

\subsection{Preprocessing}
{\scriptsize
\inputminted[linenos]{python}{./code/preprocessing.py}
\captionof{listing}{Data preprocessing code in Python \cite{zelenayOutlierDetectionWater2022}\label{listing:data-preprocessing}}}

\section{Outlier Detection Methods Implementation}
{\scriptsize
\inputminted[linenos]{python}{./code/outlier_detection_methods.py}
\captionof{listing}{Implementation of Outlier Detection Methods \cite{zelenayOutlierDetectionWater2022}\label{listing:outlier-detection-methods}}}

\section{Parameter Grid Search}
{\scriptsize
\inputminted[linenos]{python}{./code/threshold_based_outlier_detection.py}
\captionof{listing}{Threshold based outlier detection \cite{zelenayOutlierDetectionWater2022}\label{listing:threshold-based-outlier-detection}}}

\newpage
\bibliographystyle{IEEEtran}

\bibliography{bachelor_thesis}


% List of figures
\newpage
\listoffigures


% List of tables
\newpage
\listoftables
\clearpage

% List of source codes
\renewcommand\listoflistingscaption{List of source codes}
\listoflistings
\clearpage

% List of Algorithms
%\listofalgorithms
%\clearpage

% List of Abbreviations
\phantomsection
\addcontentsline{toc}{chapter}{List of Abbreviations}
\chapter*{List of Abbreviations}
\begin{acronym}%[XXXXX]
	\acro{IoT}[IoT]{Internet of Things}
	\acro{CSV}[CSV]{Comma Separated Values }
	\acro{JSON}[JSON]{Java Script Object Notation}
	\acro{MAD}[MAD]{Mean Absolute Deviation}
	\acro{LSTM}[LSTM]{Long Short-Term Memory}
	\acro{GRU}[GRU]{Gated Recurrent Unit}
	\acro{ARIMA}[ARIMA]{AutoRegressive Integrated Moving Average}
	\acro{GP}[GP]{Gaussian Process}
	\acro{HMAC}[HMAC]{Keyed-Hashing for Message Authentication}
\end{acronym}


\end{document}