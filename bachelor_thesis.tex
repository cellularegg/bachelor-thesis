\documentclass[Bachelor, BIF, english]{twbook}

\usepackage[T1]{fontenc}
% Hier kann je nach Betriebssystem eine der folgenden Optionen notwendig sein, um die Umlaute korrekt wiederzugeben:
% utf8, latin, applemac
%%\usepackage[ansinew]{inputenc}
\usepackage[utf8]{inputenc}
% Die nachfolgenden 2 Pakete stellen sonst nicht benötigte Features zur Verfügung
%%\usepackage{blindtext,dtklogos}
\usepackage{blindtext, dtk-logos}
\usepackage{cite}

%-------------------------------- for algorithms
%\usepackage{fullpage}
%\usepackage{times}
%\usepackage{fancyhdr,graphicx,amsmath,amssymb}
%\usepackage[ruled,vlined]{algorithm2e}
%\include{pythonlisting}


\usepackage{algorithm}
\usepackage[noend]{algpseudocode}
%--------------------------------

%-------------------------------- for code highlighting
\usepackage{minted}
%\usepackage{xcolor}
\definecolor{LightGray}{gray}{0.9}
\usemintedstyle{borland}
\definecolor{codegreen}{rgb}{0,0.6,0}
\definecolor{codegray}{rgb}{0.5,0.5,0.5}
\definecolor{codepurple}{rgb}{0.58,0,0.82}
\definecolor{backcolour}{rgb}{0.95,0.95,0.92}
%--------------------------------
\usepackage{physics}


%-------------------------------- for commenting
%--- https://tex.stackexchange.com/questions/9796/how-to-add-todo-notes
\usepackage{xargs}                      % Use more than one optional parameter in a new commands
%\usepackage[pdftex,dvipsnames]{xcolor}
\usepackage[colorinlistoftodos,prependcaption,textsize=tiny]{todonotes}
\newcommandx{\unsure}[2][1=]{\todo[linecolor=red,backgroundcolor=red!25,bordercolor=red,#1]{#2}}
\newcommandx{\change}[2][1=]{\todo[linecolor=blue,backgroundcolor=blue!25,bordercolor=blue,#1]{#2}}
%\newcommandx{\info}[2][1=]{\todo[linecolor=OliveGreen,backgroundcolor=OliveGreen!25,bordercolor=OliveGreen,#1]{#2}}
\newcommandx{\improvement}[2][1=]{\todo[linecolor=Plum,backgroundcolor=Plum!25,bordercolor=Plum,#1]{#2}}
%\newcommandx{\thiswillnotshow}[2][1=]{\todo[disable,#1]{#2}}
%-------------------------------- 


%-------------------------------- for UML
\usepackage[]{pgf-umlcd}

%-------------------------------- 

\title{Anomaly detection in data for data cleansing}
% \extratitle{Der Untertietel}
\author{David Zelenay}
\studentnumber{000000000000}
\supervisor{Degree First Name Surname}
\place{Vienna}
\kurzfassung{
    \todo{Change this}
    TODO Add Kurzfassung
}
\outline{
    \todo{Change this}
    This paper analyzes methods for anomaly detection to cleanse data. An overview of some key features of data quality is provided in the beginning. The difference between data cleaning and data cleansing is elaborated. Additionally a few methods for anomaly detection (mainly outlier detection) are outlined. The hypothesis of this paper is: Which characteristics define data quality, with regard to IoT (Internet of Things) Sensors? Which methods are there to detect and clean or cleanse faulty data?
}
\begin{document}
\maketitle
\chapter{Introduction}

With the growing popularity of Internet of Things (\ac{IoT}) and digitizing business processes there is a growing amount of data available for analysis.
In order to utilize the data from the IoT sensors it needs to be preprocessed. One step of preprocessing is data cleaning (also referred as data cleansing).
The main goal of data cleansing is to increase the data quality and furthermore to detect and remove anomalies in the data. The quality requirements for the data can differ depending on the use case. Anomalies in sensor data are datapoints which do not picture the reality. For example an anomaly of a temperature sensor would be if the sensor reads 0 $^{\circ}$C and the real temperature is 23 $^{\circ}$C.

\section{Research method}
This paper is a literature research. To get an overview of the topic, papers related to: data cleaning, anomaly detection for IoT data / time series data and outlier detection methods were researched. After some base knowledge was established the major topics of the paper were defined. Subsequently more research was done on the major topics (Features of data quality, data cleaning \& cleansing, outlier detection). To organize the references found while researching Zotero was used, with the Add-on Better BibTeX.

\section{Features of data quality}\label{data-quality-features}
This section will provide a few example key features of data quality. 
\subsubsection{Completeness}
Data completeness describes the wholeness of data. If there are certain aspects of data missing the data is not complete. For example if each datapoint of a sensor includes the date, time and production speed, the data is not complete, if one of those features is missing or not entire, this datapoint is not complete. \cite{caiChallengesDataQuality2015, songIoTDataQuality2020}
\subsubsection{Accuracy}
The accuracy of data describes the exactness. Example for possible data which decrease the accuracy are outliers or time shifts. Usually the accuracy of data is harder to measure than the completeness, consistency, structure or documentation. Due to the heterogeneity of sensor data (regarding numerical values like production speed or temperature, not categorical values like on/off) for each datapoint it is difficult to detect which values are genuine and which are sensor errors and therefore outliers. \cite{caiChallengesDataQuality2015}
\subsubsection{Consistency}
One example for consistency would be, if the data interval is equal. For example there should be a datapoint every ten seconds. As soon as two datapoints are more than ten seconds apart from each other the data is not consistent anymore. \cite{caiChallengesDataQuality2015}
\subsubsection{Structure \& Documentation}
If the structure of the data is not homogeneous, it is very difficult to analyze in an automated way. As a result the data either needs to be structured from the beginning or a process needs to be fabricated to structure the data automatically. Furthermore documentation is in order to structure and preprocess data. Documentation of data might include data format (CSV, parquet \cite{ApacheParquet2021}, JSON), date format (e.g. ISO 8601 with UTC offset), valid value spans (e.g. temperature is only valid if it is between 100 and 400 $^{\circ}$C)
\cite{caiChallengesDataQuality2015}

\section{Improving data quality}
This section will describe methods to improve data quality, based on the features elaborated in section \ref{data-quality-features}.
\subsubsection{Completeness}
The most common methods to increase data completeness are statistical and deep learning based approaches. The goal of these methods are to fill in the missing values of a dataset. An example for a statistical method is DynaMMo\cite{liDynaMMoMiningSummarization2009}. For ANNs (artificial neural networks) LSTM (Long short-term memory) or GRU (Gated recurrent unit) can be used to predict missing data. \cite{songIoTDataQuality2020}
\subsubsection{Accuracy}
One approach to increase the accuracy of data is to define constraints for each value. E.g. When a machine cannot produce more than ten pieces per second, because it is physically not possible, the value could be limited to less or equal than ten. However limiting the values to a specific range might hide the fact that the machine  has an error and is producing faulty products at a rate of 15 pieces per second. This is one of the reasons why more sophisticated outlier detection methods are used.\cite{songIoTDataQuality2020}

\subsubsection{Consistency}
To facilitate consistent data, statistical smoothing or forecasting methods can be used. Examples methods are: ARIMA (Autoregressive integrated moving average) or GP ( Gaussian Process). ANNs can also be used to unify the time series interval between datapoints. \cite{songIoTDataQuality2020}

\subsubsection{Structure \& Documentation}
The process of structuring heterogeneous and messy data is called data wrangling. In order to unify the structure of the data at least some documentation is required. Therefore the documentation of the data is fundamental in order to analyse or further process it. 
\chapter{Data cleaning \& cleansing approaches}
There are two main methods when it comes to data cleaning or cleansing. Ignoring faulty data or replacing it with a representative value. This paper will use the term data cleaning to describe the process of ignoring or deleting incorrect data and the term data cleansing to portray the process of replacing invalid data with representative values. Faulty, incorrect, invalid or wrong data is data which is inaccurate, incomplete or inconsistent.\\
Example sensor data: (Valid values for \verb|production_speed| range from $0.00$ to $2.00$ meter(s) per minute)
\begin{table}[ht]
\begin{tabular}{|l|l|l|l|}
\hline
\verb|ID| & \verb|timestamp|        & \verb|production_speed| (meter/minute) & \verb|machine_running| \\ \hline
$0$       & 2021-12-01T12:00:00.000 & $1.56$                                 & True                   \\ \hline
$1$       & 2021-12-01T12:01:00.000 & $1.58$                                 & True                   \\ \hline
$2$       & 2021-12-01T12:02:00.000 & $3.50$                                 & True                   \\ \hline
$3$       & 2021-12-01T12:03:00.000 & $1.50$                                 & False                  \\ \hline
$4$       & 2021-12-01T12:04:00.000 & $1.50$                                 & True                   \\ \hline
$5$       & 2021-12-01T12:05:00.000 & $1.49$                                 & True                   \\ \hline
\end{tabular}
\caption{Example of IoT sensor data}
\label{table:example-iot-data}
\end{table}
\section{Data cleaning}
As already mentioned the approach for data cleaning is to ignore or delete faulty data. Depending on the use case either the entire datapoint needs to be ignored or just one value. The process of data cleaning will be shown with the example data pictured in Table \ref{table:example-iot-data}.
The first incorrect datapoint has the \verb|ID| $2$. This row is incorrect, because the \verb|production_speed| exceeds the maximum value of $2.00$. Depending on the use case (e.g. summary of how long the machine has been running) it can make sense to just ignore the row \verb|production_speed| and keep the value for \verb|machine_running|. The second appearance of a faulty datapoint has the \verb|ID| $2$. This datapoint is incorrect since \verb|machine_running| is False but the value of \verb|production_speed| is not $0.00$. In this case it does not make sense to keep either of those values for further analysis, because it is impossible to determine which of the two columns are incorrect. A possible result after the data cleaning is shown in Table \ref{table:example-iot-data-after-cleaning}
\begin{table}[ht]
\begin{tabular}{|l|l|l|l|}
\hline
\verb|ID| & \verb|timestamp|        & \verb|production_speed| (meter/minute) & \verb|machine_running| \\ \hline
$0$       & 2021-12-01T12:00:00.000 & $1.56$                                 & True                   \\ \hline
$1$       & 2021-12-01T12:01:00.000 & $1.58$                                 & True                   \\ \hline
$2$       & 2021-12-01T12:02:00.000 &                                        & True                   \\ \hline
$3$       & 2021-12-01T12:03:00.000 &                                        &                        \\ \hline
$4$       & 2021-12-01T12:04:00.000 & $1.50$                                 & True                   \\ \hline
$5$       & 2021-12-01T12:05:00.000 & $1.49$                                 & True                   \\ \hline
\end{tabular}
\caption{Example of IoT sensor data after cleaning}
\label{table:example-iot-data-after-cleaning}
\end{table}

\section{Data cleansing}
Data cleansing pursues a different approach. Incorrect data is not ignored, but substituted by a representative value. For example for the datapoint with the \verb|ID| $2$ there are several strategies that could be followed. For example the outlier value $3.50$ could be replaced with the upper limit of the valid range, in this example $2.00$, the value could also be replaced with the last valid value, in this example $1.58$, or the value could be replaced with the average of the last $n$ Values, for example with $\frac{1.56+1.58}{2} = 1.57$. For the datapoint with the \verb|ID| $3$ there are also different approaches. Either the machine was indeed not running then it would make sense, to set the \verb|production_speed| to $0.0$, if short downtimes for this machine are very unlikely then the \verb|machine_running| value could be set to True. A possible result after the data cleansing is shown in Table \ref{table:example-iot-data-after-cleansing} \cite{maleticDataCleansingIntegrity2000}
\begin{table}[ht]
\begin{tabular}{|l|l|l|l|}
\hline
\verb|ID| & \verb|timestamp|        & \verb|production_speed| (meter/minute) & \verb|machine_running| \\ \hline
$0$       & 2021-12-01T12:00:00.000 & $1.56$                                 & True                   \\ \hline
$1$       & 2021-12-01T12:01:00.000 & $1.58$                                 & True                   \\ \hline
$2$       & 2021-12-01T12:02:00.000 & $2.00$                                 & True                   \\ \hline
$3$       & 2021-12-01T12:03:00.000 & $1.50$                                 & True                   \\ \hline
$4$       & 2021-12-01T12:04:00.000 & $1.50$                                 & True                   \\ \hline
$5$       & 2021-12-01T12:05:00.000 & $1.49$                                 & True                   \\ \hline
\end{tabular}
\caption{Example of IoT sensor data after cleansing}
\label{table:example-iot-data-after-cleansing}
\end{table}
\chapter{Outlier detection}
Outliers can be categorized as point outliers or subsequence outliers.
\subsubsection{Point outliers}
A point outlier is a single datapoint that strongly varies from the usual trend of the datapoints. \cite{blazquez-garciaReviewOutlierAnomaly2020}

\subsubsection{Subsequence outliers}
Subsequence outliers are multiple consecutive datapoints that strongly vary from the usual trend of the datapoints. \cite{blazquez-garciaReviewOutlierAnomaly2020}\\\\
Furthermore outliers can be divided into local and global outliers. 
\subsubsection{Local outliers}
A local outlier has a greater variance to its direct neighbouring datapoints (previous and next one) \cite{blazquez-garciaReviewOutlierAnomaly2020}

\subsubsection{Global outliers}
Whereas a global outlier varies more in regard to all datapoints.
To do: Add charts that visualize point, subsequence, local and global outliers
\cite{blazquez-garciaReviewOutlierAnomaly2020}

\section{Outlier detection approaches}
Outlier detection methods can be divided into the following groups
\subsubsection{Statistical}
For statistical outlier detection, historical data is taken to develop a model that pictures the expected behaviour of the data. An example of a statistical outlier detection is the threshold based method described in section \ref{threshold-based-outlier-detection} \cite{cookAnomalyDetectionIoT2020, giannoniAnomalyDetectionModels2018}

\subsubsection{Distance based}
For this approach a distance metric needs to be defined, (e.g. Euclidean distance). Then each datapoint is compared to the data preceding it. The greater the distance between the current and previous datapoints the greater the probability of an anomaly. \cite{cookAnomalyDetectionIoT2020, giannoniAnomalyDetectionModels2018, chandolaAnomalyDetectionSurvey2009}

\subsubsection{Clustering}
Clustering also requires a set of historical data in order to train the clustering model. Usually the data is clustered into two clusters: normal data and anomalous data. Depending on the distance of a new datapoint to the "normal" and the "anomalous" cluster it is classified.
\cite{cookAnomalyDetectionIoT2020, giannoniAnomalyDetectionModels2018, chandolaAnomalyDetectionSurvey2009}
\subsubsection{Predictive}
In this approach a prediction model needs to be developed, based on previous data. The prediction of this model is then compared with the actual datapoint (new data, which was not used in training the model). If the actual datapoint differs too much from the prediction it is labelled as an anomaly. \cite{cookAnomalyDetectionIoT2020, giannoniAnomalyDetectionModels2018}


\subsubsection{Ensemble}
as the word ensemble suggests, this is a collection of outlier detection methods that use a specific vote mechanism to determine whether a datapoint is faulty or normal. For example using the majority vote system and a statistical, distance based and predictive method to detect outliers. If at least two methods flag a datapoint as an outlier the ensemble reports it as an outlier as well. If only one method reports it as an outlier the ensemble does not flag it as an anomaly.
\cite{cookAnomalyDetectionIoT2020}

\section{Threshold based outlier detection} \label{threshold-based-outlier-detection}
Threshold based detection methods are able to identify outliers based on a given threshold $\tau$. These Methods can be described with the following formula
$$
|x_t - \hat{x}_t | > \tau \text{ \cite{blazquez-garciaReviewOutlierAnomaly2020}}
$$
Where $x_t$ is the actual value and $\hat{x}_t$ is the expected value and $\tau$ is a given threshold.\\
Methods to calculate $\hat{x}_t$ will be described in the following sections. Furthermore $\hat{x}_t$ can be calculated using the entire data series or with subsets (of equal length) of the entire data series. This means $\hat{x}_t$ can be either calculated for the whole data series or for just a segment.\\
Depending on the sensitivity wanted for outlier detection an appropriate $\tau$ needs to be chosen. The greater $\tau$ is the fewer outliers will be detected. The smaller $\tau$ is the more outliers will be identified.  \cite{blazquez-garciaReviewOutlierAnomaly2020}

\subsubsection{Mean}
$$
\text{mean} = \overline{x} = \frac{1}{n} \sum^n_{t=0}x_t
$$
Where $n$ is the total number of samples. Using the mean as an expected value is not robust to outliers, because the median is not as robust as the mean in hindsight to outliers. To calculate the mean all datapoints of a series must be summed up and then divided by the number of datapoints.
\subsubsection{Median}
If $n$ is odd:
$$
median(x) = x_{(n+1)/2}
$$
If $n$ is even:
$$
median(x) = \frac{x_{n/2} + x_{(n+1)/2}}{2}
$$
Where $x$ is a dataset of $n$ elements ordered from smallest to largest\\
($x_1 \leq x_2 \leq x_3 \leq \ldots \leq x_{n-2} \leq x_{n-1} \leq x_n$)
\cite{blazquez-garciaReviewOutlierAnomaly2020}
To calculate the median all values must be sorted from smallest to largest. If the number of datapoints is odd then the most center datapoint is the Median (e.g. if the series consist of 7 values the third value is the median). If the number of datapoints is even then the median is the mean of the two datapoints in the center.
\subsubsection{Median Absolute Deviation (\ac{MAD})}
The Median Absolute Deviation 
$$
MAD = median(|x_t - median(x)|)
$$
$MAD$ is a more robust (regarding outliers) way to calculate the deviation of a dataset. To calculate the $MAD$ firstly the median of the dataset must be calculated. Then the absolute difference between $x_t$ and the median of the dataset is calculated. The Median of all differences results in the $MAD$
\cite{leysDetectingOutliersNot2013, mehrangOutlierDetectionWeight2015}






\chapter{Conclusion}
This bachelor thesis provides an overview of the topic anomaly detection, especially on outlier detection for time series data. It provides a description for key features of data quality, and introduces the topic of data cleaning and data cleansing. Furthermore this paper provides general overview of outlier detection approaches. After a theoretical overview of different outlier detection approaches they are tested on water level data from different rivers.
\par
% There are countless methods to detect anomalies in data. There is not a go-to approach that suits all needs. It is required to assess different approaches for different applications, in order to get the best result. This paper should provide an overview of approaches to detect outliers / anomalies. It depends on the use case which method to detect outliers has the highest success rate.  
The overall bes performance, across different water level measurement stations, was achieved by using the median threshold based outlier detection method with a centered window with the size of three and a threshold of about 6.6. The median threshold based outlier detection also delivered the hightest $F_1-score\;(0.905)$. Using the mean to calculate $\hat{x}_t$ is not recommended since the mean is not robust against outliers. Using the \ac{MAD} with the threshold based outlier detection resulted in the lowest $F_1-score$, with the best score only being about 0.45. The second best result was achieved by using the modified z-score described in \autoref{section:outlier-detection-modified-z-score}. For the stations tested the approach using the median delivered the best performances. However this does not mean, that this will be true for all stations. It has to be assessed for each station individually which model is able to detect best. Furthermore it depends on the use case if higher precision or recall is required. Depending on that, $\beta$ for the $F_{\beta}-score$ needs to be chosen accordingly. For the tests the $F_1-score$ was used since precision and recall are equally important. In addition preprocessing the data by setting upper and lower boundaries and removing datapoints which exceed those limits, did not improve the performance of the models, on the contrary the performance was worse. Because the extreme outliers were mostly detected anyways, thus fewer outliers were detected when setting upper and lower limits, which resulted in a lower performance.

\change{Ich schlage hier vor auch einen Teil der quanitativen Analyse zusammenzufassen. Dabei sollten die Methoden und die Ergebnisse kurz skizziert werden. Vielleicht lassen sich auch Entscheidungen ableiten wann welche Methode besser greift.}
% \todo{Change this! Currently copied from the paper.}
% \section{Advantages and Disadvantages of used Outlier Detection Methods}
% \todo{write}
\chapter{Future Work}
Chapter about which additional approaches could be tested:
\begin{itemize}
    \item setting a maximum gradient for both directions (one for rising and falling values) for each measurement station.
    \item \acp{ANN} with \ac{LSTM} or \ac{GRU} maybe also autoencoder architecture
    \item Prediction vs classification \ac{ANN}
    \item 1.5 times the \ac{IQR}
\end{itemize} 
\todo{Should I also include this or is this not common for a bachelor thesis? -> YES}
\change{Die Methode wollte ich auch vorschlagen, aber der Umfang ist so schon gross genug.}
\newpage
\bibliographystyle{IEEEtran}

\bibliography{bachelor_thesis}


% List of figures
\newpage
\listoffigures


% List of tables
\newpage
\listoftables
\clearpage

% List of source codes
\renewcommand\listoflistingscaption{List of source codes}
\listoflistings
\clearpage

% List of Algorithms
%\listofalgorithms
%\clearpage

% List of Abbreviations
\phantomsection
\addcontentsline{toc}{chapter}{List of Abbreviations}
\chapter*{List of Abbreviations}
\begin{acronym}%[XXXXX]
	\acro{IoT}[IoT]{Internet of Things}
	\acro{CSV}[CSV]{Comma Separated Values }
	\acro{JSON}[JSON]{Java Script Object Notation}
	\acro{MAD}[MAD]{Mean Absolute Deviation}
	\acro{LSTM}[LSTM]{Long Short-Term Memory}
	\acro{GRU}[GRU]{Gated Recurrent Unit}
	\acro{ARIMA}[ARIMA]{AutoRegressive Integrated Moving Average}
	\acro{GP}[GP]{Gaussian Process}
	\acro{HMAC}[HMAC]{Keyed-Hashing for Message Authentication}
\end{acronym}


\end{document}