\chapter{Overview}

\section{Introduction}\label{section:introduction}
With the growing popularity of \ac{IoT} and digitizing business processes there is a growing amount of data available for analysis.
In order to utilize the data from the \ac{IoT} sensors it needs to be preprocessed. One step of preprocessing is data cleaning (also referred as data cleansing).
The main goal of data cleansing is to increase the data quality and furthermore to detect and remove anomalies in the data. The quality requirements for the data can differ depending on the use case. Anomalies in sensor data are datapoints which do not picture the reality. For example an anomaly of a temperature sensor would be if the sensor reads 0 $^{\circ}$C and the real temperature is 23 $^{\circ}$C. This paper specifically focuses on the removal of outliers for water level sensors. The data is provided by FloodAlert\cite{strassmayrFloodAlertWaterLevels}, which provides a service to warn people about floods, for their area of interest.
% \todo{Update Introduction}

\section{Research Question}
What are common methods to detect outliers for time series data? \newline
How can outlier detection methods be compared, with a focus on water level data? \newline
Based on water levels from different rivers, which method is able to classify outliers most reliably?
\todo{Add story to research questions}


\section{Research Method}
This thesis will provide an overview and comparison of different approaches to detect outliers. It will focus on time series data, especially water level measurements of rivers. To introduce the topic a general overview about data quality, data cleansing / cleaning and outlier types is provided. For the theoretical pars of the chapters literature research was conducted. After gathering knowledge on different outlier detection approaches they were implemented in Python. To test the performance a suitable performance metric needed to be chosen. To use real world data to classify outliers, the water levels from different measurement stations were taken. In the end the performance of the different approaches to detect outliers are compared.
% This paper is a literature research. To get an overview of the topic, papers related to: data cleaning, anomaly detection for IoT data / time series data and outlier detection methods were researched. After some base knowledge was established the major topics of the paper were defined. Subsequently more research was done on the major topics (Features of data quality, data cleaning \& cleansing, outlier detection). To organize the references found while researching Zotero was used, with the Add-on Better BibTeX. \todo{Adapt research method}
