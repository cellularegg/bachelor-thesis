\chapter{Data cleaning \& cleansing approaches}
There are two main methods when it comes to data cleaning or cleansing. Ignoring faulty data or replacing it with a representative value. This paper will use the term data cleaning to describe the process of ignoring or deleting incorrect data and the term data cleansing to portray the process of replacing invalid data with representative values. Faulty, incorrect, invalid or wrong data is data which is inaccurate, incomplete or inconsistent.\\
Example sensor data: (Valid values for \verb|production_speed| range from $0.00$ to $2.00$ meter(s) per minute)
\begin{table}[ht]
\begin{tabular}{|l|l|l|l|}
\hline
\verb|ID| & \verb|timestamp|        & \verb|production_speed| (meter/minute) & \verb|machine_running| \\ \hline
$0$       & 2021-12-01T12:00:00.000 & $1.56$                                 & True                   \\ \hline
$1$       & 2021-12-01T12:01:00.000 & $1.58$                                 & True                   \\ \hline
$2$       & 2021-12-01T12:02:00.000 & $3.50$                                 & True                   \\ \hline
$3$       & 2021-12-01T12:03:00.000 & $1.50$                                 & False                  \\ \hline
$4$       & 2021-12-01T12:04:00.000 & $1.50$                                 & True                   \\ \hline
$5$       & 2021-12-01T12:05:00.000 & $1.49$                                 & True                   \\ \hline
\end{tabular}
\caption{Example of IoT sensor data}
\label{table:example-iot-data}
\end{table}
\section{Data cleaning}
As already mentioned the approach for data cleaning is to ignore or delete faulty data. Depending on the use case either the entire datapoint needs to be ignored or just one value. The process of data cleaning will be shown with the example data pictured in Table \ref{table:example-iot-data}.
The first incorrect datapoint has the \verb|ID| $2$. This row is incorrect, because the \verb|production_speed| exceeds the maximum value of $2.00$. Depending on the use case (e.g. summary of how long the machine has been running) it can make sense to just ignore the row \verb|production_speed| and keep the value for \verb|machine_running|. The second appearance of a faulty datapoint has the \verb|ID| $2$. This datapoint is incorrect since \verb|machine_running| is False but the value of \verb|production_speed| is not $0.00$. In this case it does not make sense to keep either of those values for further analysis, because it is impossible to determine which of the two columns are incorrect. A possible result after the data cleaning is shown in Table \ref{table:example-iot-data-after-cleaning}
\begin{table}[ht]
\begin{tabular}{|l|l|l|l|}
\hline
\verb|ID| & \verb|timestamp|        & \verb|production_speed| (meter/minute) & \verb|machine_running| \\ \hline
$0$       & 2021-12-01T12:00:00.000 & $1.56$                                 & True                   \\ \hline
$1$       & 2021-12-01T12:01:00.000 & $1.58$                                 & True                   \\ \hline
$2$       & 2021-12-01T12:02:00.000 &                                        & True                   \\ \hline
$3$       & 2021-12-01T12:03:00.000 &                                        &                        \\ \hline
$4$       & 2021-12-01T12:04:00.000 & $1.50$                                 & True                   \\ \hline
$5$       & 2021-12-01T12:05:00.000 & $1.49$                                 & True                   \\ \hline
\end{tabular}
\caption{Example of IoT sensor data after cleaning}
\label{table:example-iot-data-after-cleaning}
\end{table}

\section{Data cleansing}
Data cleansing pursues a different approach. Incorrect data is not ignored, but substituted by a representative value. For example for the datapoint with the \verb|ID| $2$ there are several strategies that could be followed. For example the outlier value $3.50$ could be replaced with the upper limit of the valid range, in this example $2.00$, the value could also be replaced with the last valid value, in this example $1.58$, or the value could be replaced with the average of the last $n$ Values, for example with $\frac{1.56+1.58}{2} = 1.57$. For the datapoint with the \verb|ID| $3$ there are also different approaches. Either the machine was indeed not running then it would make sense, to set the \verb|production_speed| to $0.0$, if short downtimes for this machine are very unlikely then the \verb|machine_running| value could be set to True. A possible result after the data cleansing is shown in Table \ref{table:example-iot-data-after-cleansing} \cite{maleticDataCleansingIntegrity2000}
\begin{table}[ht]
\begin{tabular}{|l|l|l|l|}
\hline
\verb|ID| & \verb|timestamp|        & \verb|production_speed| (meter/minute) & \verb|machine_running| \\ \hline
$0$       & 2021-12-01T12:00:00.000 & $1.56$                                 & True                   \\ \hline
$1$       & 2021-12-01T12:01:00.000 & $1.58$                                 & True                   \\ \hline
$2$       & 2021-12-01T12:02:00.000 & $2.00$                                 & True                   \\ \hline
$3$       & 2021-12-01T12:03:00.000 & $1.50$                                 & True                   \\ \hline
$4$       & 2021-12-01T12:04:00.000 & $1.50$                                 & True                   \\ \hline
$5$       & 2021-12-01T12:05:00.000 & $1.49$                                 & True                   \\ \hline
\end{tabular}
\caption{Example of IoT sensor data after cleansing}
\label{table:example-iot-data-after-cleansing}
\end{table}